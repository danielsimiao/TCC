\chapter{Introdução}



\section{Motivação}


O posicionamento social está mudando e junto com ele a forma de consumir produtos e serviços sustentáveis. Esta postura, contribui para a preservação do planeta, uma vez que os recursos renováveis são limitados, forçando a humanidade trilhar um caminho mais consciente e harmonioso com a natureza. Assim, o interesse em discorrer sobre a geração de eletricidade utilizando o biogás do aterro sanitário, surgiu a partir da necessidade de desenvolver projetos sustentáveis com foco na geração de energia limpa. 


\section{Justificativa}

Diante dessa problemática, surgiu o seguinte questionamento: como gerar energia elétrica utilizando os resíduos sólidos, transformando o lixo de passivo ambiental para ativo financeiro.

Petrolina (PE) está inserida no mercado internacional, com foco no escoamento da produção agrícola. Atualmente, é cada vez mais comum os consumidores estrangeiros buscarem informações completas que permitem o rastreio da cadeia produtiva de um alimento, dispondo de dados da colheita, adubação e impactos ambientais \cite{europa}. Assim, a busca por informações que dizem a real origem dos alimentos tem crescido nos últimos anos, principalmente aqueles que serão colocados na mesa do consumidor europeu. 

Esta situação evidencia a forte tendência global não só pela forma como são produzidos os alimentos, mas de onde eles vêm e sua influência socioeconômica. Uma pesquisa realizada em Bruxelas, pela Organização Europeia do Consumidor (BEUC), revelou que 70\% das pessoas entrevistadas afirmaram que conhecer a origem da comida é um fator crucial para determinar as compras no supermercado, principalmente daqueles itens consumidos diariamente \cite[pág. 6]{beuc:consumo}. Tal cenário sinaliza com luz amarela a necessidade de investir em empreendimentos que promovam a redução dos gases de efeito estufa (GEE).

As mudanças climáticas forçaram os governos a planejarem incentivos fiscais e subsídios financeiros para as fontes renováveis. São ações que proporcionam uma política de diversificação das matrizes energéticas e, consequentemente, o amadurecimento tecnológico das indústrias de energia solar, eólica e de biogás. O projeto de lei do Senado Federal nº 712/2015 aumenta a participação das energias renováveis na matriz energética em pelo menos 60\% até 2040. %desfecho

A Portaria nº 65 do Ministério de Minas e Energia (MME), publicada no dia 28 de fevereiro de 2018, no Diário Oficial da União, determinou um Valor Anual de Referência Específico (VRES) para a biomassa dedicada de R\$ 537,00 e gás natural a um custo de R\$ 451,00 por MWh. A valorização do MWh para fornecimento de eletricidade, com energias renováveis, representa um cenário muito atrativo para investimento em usinas termoelétricas que utilizam o biogás como combustível.


\section{Objetivo Geral}
Identificar a viabilidade econômica da construção de uma usina termoelétrica, localizada no aterro sanitário de Petrolina (PE), aproveitando o biogás gerado pelo lixo em decomposição.


\section{Objetivos Específicos}

\begin{itemize}
    \item Analisar o tempo de vida útil e a capacidade de produção de energia elétrica do aterro sanitário;
    \item Identificar o perfil financeiro e o consumo de energia elétrica do município ao longo do ano de 2018;
    \item Avaliar o tempo de retorno do investimento e ganhos financeiros para a administração pública.
\end{itemize}

\section{A Relevância do Trabalho}
O presente trabalho é importante porque promove exploração energética do metano que é produzido pelo aterro sanitário. Bem como, é uma atividade econômica inexplorada na região, cenário que condiciona a evolução ou criação de empresas locais voltadas para o reaproveitamento energético do lixo urbano.

\section{Organização do Trabalho}
Este trabalho é composto por cinco capítulos, os quais são descritos a seguir:

\begin{itemize}
    \item No Capítulo 2, são apresentados os fundamentos teóricos quanto às formas de produção, conversão e transmissão de energia, além da estruturação do sistema elétrico brasileiro voltado para o biometano, uma evolução no aproveitamento das fontes renováveis ante o uso de combustíveis fósseis;
    \item No Capítulo 3, apresenta-se uma visão micro e macroeconômica da cidade de Petrolina (PE) no contexto da globalização;
    \item No Capítulo 4, descreve-se a metodologia utilizada para projetar uma usina termelétrica em minigeração distribuída, considerando a localização e potência desejada
    \item No Capítulo 5, são apresentados os resultados pertinentes à rentabilidade, financiamento, compensação ou venda da eletricidade, a fim de comprovar a viabilidade econômica do empreendimento;
    \item No Capítulo 6, são realizadas as considerações finais, focalizando as reflexões sobre a geração de energia elétrica com biogás no aterro sanitário de Petrolina (PE).
\end{itemize}
